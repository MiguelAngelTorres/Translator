\documentclass[twoside]{book}

% Packages required by doxygen
\usepackage{calc}
\usepackage{doxygen}
\usepackage{graphicx}
\usepackage[utf8]{inputenc}
\usepackage{makeidx}
\usepackage{multicol}
\usepackage{multirow}
\usepackage{textcomp}
\usepackage[table]{xcolor}

% NLS support packages
\usepackage[spanish]{babel}
% Font selection
\usepackage[T1]{fontenc}
\usepackage{mathptmx}
\usepackage[scaled=.90]{helvet}
\usepackage{courier}
\usepackage{amssymb}
\usepackage{sectsty}
\renewcommand{\familydefault}{\sfdefault}
\allsectionsfont{%
  \fontseries{bc}\selectfont%
  \color{darkgray}%
}
\renewcommand{\DoxyLabelFont}{%
  \fontseries{bc}\selectfont%
  \color{darkgray}%
}

% Page & text layout
\usepackage{geometry}
\geometry{%
  a4paper,%
  top=2.5cm,%
  bottom=2.5cm,%
  left=2.5cm,%
  right=2.5cm%
}
\tolerance=750
\hfuzz=15pt
\hbadness=750
\setlength{\emergencystretch}{15pt}
\setlength{\parindent}{0cm}
\setlength{\parskip}{0.2cm}
\makeatletter
\renewcommand{\paragraph}{%
  \@startsection{paragraph}{4}{0ex}{-1.0ex}{1.0ex}{%
    \normalfont\normalsize\bfseries\SS@parafont%
  }%
}
\renewcommand{\subparagraph}{%
  \@startsection{subparagraph}{5}{0ex}{-1.0ex}{1.0ex}{%
    \normalfont\normalsize\bfseries\SS@subparafont%
  }%
}
\makeatother

% Headers & footers
\usepackage{fancyhdr}
\pagestyle{fancyplain}
\fancyhead[LE]{\fancyplain{}{\bfseries\thepage}}
\fancyhead[CE]{\fancyplain{}{}}
\fancyhead[RE]{\fancyplain{}{\bfseries\leftmark}}
\fancyhead[LO]{\fancyplain{}{\bfseries\rightmark}}
\fancyhead[CO]{\fancyplain{}{}}
\fancyhead[RO]{\fancyplain{}{\bfseries\thepage}}
\fancyfoot[LE]{\fancyplain{}{}}
\fancyfoot[CE]{\fancyplain{}{}}
\fancyfoot[RE]{\fancyplain{}{\bfseries\scriptsize Generado el Domingo, 1 de Noviembre de 2015 09\-:27\-:59 para Traductor por Doxygen }}
\fancyfoot[LO]{\fancyplain{}{\bfseries\scriptsize Generado el Domingo, 1 de Noviembre de 2015 09\-:27\-:59 para Traductor por Doxygen }}
\fancyfoot[CO]{\fancyplain{}{}}
\fancyfoot[RO]{\fancyplain{}{}}
\renewcommand{\footrulewidth}{0.4pt}
\renewcommand{\chaptermark}[1]{%
  \markboth{#1}{}%
}
\renewcommand{\sectionmark}[1]{%
  \markright{\thesection\ #1}%
}

% Indices & bibliography
\usepackage{natbib}
\usepackage[titles]{tocloft}
\setcounter{tocdepth}{3}
\setcounter{secnumdepth}{5}
\makeindex

% Hyperlinks (required, but should be loaded last)
\usepackage{ifpdf}
\ifpdf
  \usepackage[pdftex,pagebackref=true]{hyperref}
\else
  \usepackage[ps2pdf,pagebackref=true]{hyperref}
\fi
\hypersetup{%
  colorlinks=true,%
  linkcolor=blue,%
  citecolor=blue,%
  unicode%
}

% Custom commands
\newcommand{\clearemptydoublepage}{%
  \newpage{\pagestyle{empty}\cleardoublepage}%
}


%===== C O N T E N T S =====

\begin{document}

% Titlepage & ToC
\hypersetup{pageanchor=false}
\pagenumbering{roman}
\begin{titlepage}
\vspace*{7cm}
\begin{center}%
{\Large Traductor }\\
\vspace*{1cm}
{\large Generado por Doxygen 1.8.6}\\
\vspace*{0.5cm}
{\small Domingo, 1 de Noviembre de 2015 09:27:59}\\
\end{center}
\end{titlepage}
\clearemptydoublepage
\tableofcontents
\clearemptydoublepage
\pagenumbering{arabic}
\hypersetup{pageanchor=true}

%--- Begin generated contents ---
\chapter{Pagina Principal del Proyecto Traductor}
\label{index}\hypertarget{index}{}  \hypertarget{index_intro_sec}{}\section{Introducción}\label{index_intro_sec}
Este es un proyecto desarrollado con el fin de simplificar la búsqueda de traducciones de un idioma a otro. Esta pensado para un uso personal y permite la versatilidad que un cuaderno común no posee, ya que permite buscar palabras y sus posibles traducciones a otro idioma de manera rápida y sencilla.\hypertarget{index_use_sec}{}\section{Uso}\label{index_use_sec}
\hypertarget{index_step1}{}\subsection{Paso 1\-: Creación de una libreta}\label{index_step1}
Con el fin de almacenar tantas palabras con sus respectivas traducciones como sea posible, será necesario crear un fichero con la extension .txt que almacene las palabras con sus traducciones, debe tener el siguiente formato\-:


\begin{DoxyCodeInclude}
Adan;Adam;
Alcoran;Koran;
Alejandro;Alexander;
Alemania;Germany;
Alpes;Alps;
Amberes;Antwerp;
Amsterdam;Amsterdam;
America;America;
America del Sur;SouthAmerica;
America meridional;SouthAmerica;
America septentrional;NorthAmerica;
Andalucia;Andalusia;
Andorra;Andorra;
Antillas;Antilles;
Anunciacion;Annunciation;Lady Day;
Apocalipsis;Apocalypse;
Aquisgran;Aix‐la‐Chapelle;
Arabia;Arabia;
Argel;Algiers;
Argentina;Argentina;
Arizona;Arizona;
Aruba;Aruba;
Ascension;AscensionDay;
Asia;Asia;
Asia Menor;AsiaMinor;
Asturias;Asturias;
Atenas;Athens;
Australia;Australia;
Austria;Austria;
Balkanes;Balkans;
Barcelona;Barcelona;
Bengala;Bengal;
Berlin;Berlin;
Biblia;Bible;
Blanca‐Nieve;Little Snow White;
Bolivia;Bolivia;
Brasil;Brazil;
Bruselas;Brussels;
Bulgaria;Bulgaria;
Burdeos;Bordeaux;
Baltico;BalticSea;
Belgica;Belgium;
Canada;Canada;
Candelaria;Candlemas;
Castilla;Castile;
Cataluna;Catalonia;
Cenicienta;Cinderella;
Cenopegias;Feast of Tabernacles;
Cerdena;Sardinia;Sardina;
Checoeslovaquia;Czechoslovakia;
Chile;Chile;
China;China;
Colombia;Colombia;
Colonia;Cologne;
Colon;Columbus;
Costa Brava;Costa Brava;
Costa Rica;Costa Rica;
Creta;Crete;
Cristo;Christ;
Cruz Roja;Red Cross;
Cuba;Cuba;
Curazao;Curaçao;
Caucaso;Caucasus;
Corcega;Corsica;
Cordoba;Cordova;
Dinamarca;Denmark;
Dios;God;
Ecuador;Ecuador;
Egipto;Egypt;
El Salvador;El Salvador;
Escalda;Scheldt;
Escandinavia;Scandinavia;
Escocia;Scotland;
Espana;Spain;
Estados Unidos;United States of America;USA;
Estonia;Estonia;

\end{DoxyCodeInclude}
 S\-I\-E\-N\-D\-O E\-L Ú\-L\-T\-I\-M\-O \par
 I\-M\-P\-R\-E\-S\-C\-I\-N\-D\-I\-B\-L\-E P\-A\-R\-A Q\-U\-E E\-L P\-R\-O\-G\-R\-A\-M\-A F\-U\-N\-C\-I\-O\-N\-E C\-O\-R\-R\-E\-C\-T\-A\-M\-E\-N\-T\-E\hypertarget{index_step2}{}\subsection{Paso 2\-: Búsqueda de traducciones}\label{index_step2}
Una vez creado el fichero libreta, solo será necesario cargarla tal y como se muestra en el ejecutable de prueba que viene al final de este paso. Ahora cuando hayas aprendido muchas palabras en un idioma y no te acuerdes de una en concreto, no será necesario recorrer una y otra vez tu libreta buscandola, basta con ejecutar tu programa y escribir la palabra deseada.


\begin{DoxyCodeInclude}
\textcolor{preprocessor}{#include "\hyperlink{traductor_8h}{traductor.h}"}
\textcolor{preprocessor}{#include "\hyperlink{palabra_8h}{palabra.h}"}
\textcolor{preprocessor}{#include <fstream>}
\textcolor{preprocessor}{#include <iostream>}
\textcolor{keyword}{using namespace }std;
\textcolor{keywordtype}{int} main(\textcolor{keywordtype}{int} argc, \textcolor{keywordtype}{char} * argv[])\{

  \textcolor{keywordflow}{if} (argc!=2)\{
      cout<<\textcolor{stringliteral}{"Dime el nombre del fichero con las traducciones"}<<endl;
      \textcolor{keywordflow}{return} 0;
   \}

   ifstream f (argv[1]);
   \textcolor{keywordflow}{if} (!f)\{
    cout<<\textcolor{stringliteral}{"No puedo abrir el fichero "}<<argv[1]<<endl;
    \textcolor{keywordflow}{return} 0;
   \}
   cout << \textcolor{stringliteral}{"voy a crear traductor"}<< endl ;
   \hyperlink{classtraductor}{traductor} T;
   cout << \textcolor{stringliteral}{"voy a cargar el traductor"}<< endl ;
   f>>T; \textcolor{comment}{//Cargamos en memoria, en el traductor.}
   
   \textcolor{keywordtype}{string} a;
   cout<<\textcolor{stringliteral}{"Dime una palabra en el idioma origen:"};
   getline(cin,a);
   
   \textcolor{keywordtype}{int} nacep=0;
  \textcolor{keywordtype}{string}* trads=T.GetTraduccion(a,nacep);
   
   
   \textcolor{comment}{/* Escrbimos*/}
   cout<<a<<\textcolor{stringliteral}{"-->"};
   \textcolor{keywordflow}{for} (\textcolor{keywordtype}{int} i=0;i<nacep; ++i)
     cout<<trads[i]<<\textcolor{charliteral}{';'};
   cout<<endl;
   
   
   



\}
\end{DoxyCodeInclude}
 
\chapter{Rep del T\-D\-A Palabra}
\label{repConjuntoPalabra}
\hypertarget{repConjuntoPalabra}{}
\hypertarget{repConjuntoTraductor_invConjunto}{}\section{Invariante de la representación}\label{repConjuntoTraductor_invConjunto}
El invariante es {\itshape rep.\-termino}\mbox{[}0\mbox{]}!=\char`\"{}\char`\"{} y {\itshape rep.\-termino}\mbox{[}1\mbox{]}!=\char`\"{}\char`\"{}\hypertarget{repConjuntoTraductor_faConjunto}{}\section{Función de abstracción}\label{repConjuntoTraductor_faConjunto}
Un objeto válido {\itshape rep} del T\-D\-A palabra representa los términos

rep.\-termino\mbox{[}0-\/rep.\-n\mbox{]} 
\chapter{Rep del T\-D\-A Traductor}
\label{repConjuntoTraductor}
\hypertarget{repConjuntoTraductor}{}
\hypertarget{repConjuntoTraductor_invConjunto}{}\section{Invariante de la representación}\label{repConjuntoTraductor_invConjunto}
El invariante es {\itshape rep.\-entradas!=N\-U\-L\-L} \hypertarget{repConjuntoTraductor_faConjunto}{}\section{Función de abstracción}\label{repConjuntoTraductor_faConjunto}
Un objeto válido {\itshape rep} del T\-D\-A Traductor representa las palabras

rep.\-entradas\mbox{[}0-\/rep.\-nentr\mbox{]} 
\chapter{Índice de clases}
\section{Lista de clases}
Lista de las clases, estructuras, uniones e interfaces con una breve descripción\-:\begin{DoxyCompactList}
\item\contentsline{section}{\hyperlink{classpalabra}{palabra} \\*T.\-D.\-A. Palabra }{\pageref{classpalabra}}{}
\item\contentsline{section}{\hyperlink{classtraductor}{traductor} \\*T.\-D.\-A. Traductor }{\pageref{classtraductor}}{}
\end{DoxyCompactList}

\chapter{Indice de archivos}
\section{Lista de archivos}
Lista de todos los archivos documentados y con descripciones breves\-:\begin{DoxyCompactList}
\item\contentsline{section}{include/\hyperlink{palabra_8h}{palabra.\-h} \\*Fichero cabecera del T\-D\-A palabra }{\pageref{palabra_8h}}{}
\item\contentsline{section}{include/\hyperlink{traductor_8h}{traductor.\-h} \\*Fichero cabecera del T\-D\-A Traductor }{\pageref{traductor_8h}}{}
\end{DoxyCompactList}

\chapter{Documentación de las clases}
\hypertarget{classpalabra}{\section{Referencia de la Clase palabra}
\label{classpalabra}\index{palabra@{palabra}}
}


T.\-D.\-A. Palabra.  




{\ttfamily \#include $<$palabra.\-h$>$}

\subsection*{Métodos públicos}
\begin{DoxyCompactItemize}
\item 
\hyperlink{classpalabra_aa1f51124f5ada68f41a1579bc9216254}{palabra} ()
\begin{DoxyCompactList}\small\item\em Constructor del T\-D\-A palabra. \end{DoxyCompactList}\item 
\hyperlink{classpalabra_a890abfacbaca2b42bb9f49e3d3182f9a}{palabra} (const \hyperlink{classpalabra}{palabra} \&w)
\begin{DoxyCompactList}\small\item\em Constructor de copia del T\-D\-A palabra. \end{DoxyCompactList}\item 
void \hyperlink{classpalabra_a4ccd489a1acdc6b11f52611562ad4af4}{cargar} (const string d)
\begin{DoxyCompactList}\small\item\em Cargar una palabra con información. \end{DoxyCompactList}\item 
\hyperlink{classpalabra_ae4c8a87d7bfa6cebf5941d4938da661e}{$\sim$palabra} ()
\begin{DoxyCompactList}\small\item\em Destructor del T\-D\-A Palabra. \end{DoxyCompactList}\item 
string $\ast$ \hyperlink{classpalabra_acef4f5bd0630c89975775218daf913bd}{Terminos} ()
\begin{DoxyCompactList}\small\item\em Devolución de terminos. \end{DoxyCompactList}\item 
const int \hyperlink{classpalabra_a9b194a7cce2d3713fa278504f09aca9c}{Tamanio} ()
\begin{DoxyCompactList}\small\item\em Tamaño de la palabra. \end{DoxyCompactList}\item 
\hyperlink{classpalabra}{palabra} \& \hyperlink{classpalabra_a9e54ed0b6c965692b48f72b41dc2a5e7}{operator=} (const \hyperlink{classpalabra}{palabra} \&original)
\begin{DoxyCompactList}\small\item\em Operador de asignación del T\-D\-A palabra. \end{DoxyCompactList}\end{DoxyCompactItemize}


\subsection{Descripción detallada}
T.\-D.\-A. Palabra. 

Una instancia {\itshape w} del tipo de datos abstracto {\ttfamily palabra} es un objeto que representa un término en un idioma de partida y varias traducciones posibles de dicho término en otro idioma.

\begin{DoxyAuthor}{Autor}
Miguel Ángel Torres 
\end{DoxyAuthor}
\begin{DoxyDate}{Fecha}
Octubre 2015 
\end{DoxyDate}


Definición en la línea 23 del archivo palabra.\-h.



\subsection{Documentación del constructor y destructor}
\hypertarget{classpalabra_aa1f51124f5ada68f41a1579bc9216254}{\index{palabra@{palabra}!palabra@{palabra}}
\index{palabra@{palabra}!palabra@{palabra}}
\subsubsection[{palabra}]{\setlength{\rightskip}{0pt plus 5cm}palabra\-::palabra (
\begin{DoxyParamCaption}
{}
\end{DoxyParamCaption}
)}}\label{classpalabra_aa1f51124f5ada68f41a1579bc9216254}


Constructor del T\-D\-A palabra. 

\begin{DoxyReturn}{Devuelve}
Crea un objeto de tipo palabra con sus campos vacios, luego es necesario cargar la palabra o el objeto estará vació. 
\end{DoxyReturn}
\hypertarget{classpalabra_a890abfacbaca2b42bb9f49e3d3182f9a}{\index{palabra@{palabra}!palabra@{palabra}}
\index{palabra@{palabra}!palabra@{palabra}}
\subsubsection[{palabra}]{\setlength{\rightskip}{0pt plus 5cm}palabra\-::palabra (
\begin{DoxyParamCaption}
\item[{const {\bf palabra} \&}]{w}
\end{DoxyParamCaption}
)}}\label{classpalabra_a890abfacbaca2b42bb9f49e3d3182f9a}


Constructor de copia del T\-D\-A palabra. 


\begin{DoxyParams}{Parámetros}
{\em w} & Palabra que va a ser copiada \\
\hline
\end{DoxyParams}
\begin{DoxyReturn}{Devuelve}
Devuelve una nueva palabra con una copia de los campos de w 
\end{DoxyReturn}
\hypertarget{classpalabra_ae4c8a87d7bfa6cebf5941d4938da661e}{\index{palabra@{palabra}!$\sim$palabra@{$\sim$palabra}}
\index{$\sim$palabra@{$\sim$palabra}!palabra@{palabra}}
\subsubsection[{$\sim$palabra}]{\setlength{\rightskip}{0pt plus 5cm}palabra\-::$\sim$palabra (
\begin{DoxyParamCaption}
{}
\end{DoxyParamCaption}
)}}\label{classpalabra_ae4c8a87d7bfa6cebf5941d4938da661e}


Destructor del T\-D\-A Palabra. 

\begin{DoxyReturn}{Devuelve}
Destruye el objeto Palabra 
\end{DoxyReturn}


\subsection{Documentación de las funciones miembro}
\hypertarget{classpalabra_a4ccd489a1acdc6b11f52611562ad4af4}{\index{palabra@{palabra}!cargar@{cargar}}
\index{cargar@{cargar}!palabra@{palabra}}
\subsubsection[{cargar}]{\setlength{\rightskip}{0pt plus 5cm}void palabra\-::cargar (
\begin{DoxyParamCaption}
\item[{const string}]{d}
\end{DoxyParamCaption}
)}}\label{classpalabra_a4ccd489a1acdc6b11f52611562ad4af4}


Cargar una palabra con información. 


\begin{DoxyParams}{Parámetros}
{\em d} & Cadena de términos con un orden especifico \\
\hline
\end{DoxyParams}
\begin{DoxyNote}{Nota}
M\-I\-R\-E P\-R\-E\-C\-O\-N\-D\-I\-C\-I\-O\-N D\-E L\-A F\-U\-N\-C\-I\-O\-N 
\end{DoxyNote}
\begin{DoxyPrecond}{Precondición}
La cadena deberá estar estructurada de la siguiente forma \-: palabra\-\_\-origen;traducción;traducción;traducción.... 
\end{DoxyPrecond}
\hypertarget{classpalabra_a9e54ed0b6c965692b48f72b41dc2a5e7}{\index{palabra@{palabra}!operator=@{operator=}}
\index{operator=@{operator=}!palabra@{palabra}}
\subsubsection[{operator=}]{\setlength{\rightskip}{0pt plus 5cm}{\bf palabra}\& palabra\-::operator= (
\begin{DoxyParamCaption}
\item[{const {\bf palabra} \&}]{original}
\end{DoxyParamCaption}
)}}\label{classpalabra_a9e54ed0b6c965692b48f72b41dc2a5e7}


Operador de asignación del T\-D\-A palabra. 


\begin{DoxyParams}{Parámetros}
{\em original} & Palabra que va a ser copiada \\
\hline
\end{DoxyParams}
\begin{DoxyReturn}{Devuelve}
Devuelve la nueva palabra para poder encadenar asignaciones 
\end{DoxyReturn}
\hypertarget{classpalabra_a9b194a7cce2d3713fa278504f09aca9c}{\index{palabra@{palabra}!Tamanio@{Tamanio}}
\index{Tamanio@{Tamanio}!palabra@{palabra}}
\subsubsection[{Tamanio}]{\setlength{\rightskip}{0pt plus 5cm}const int palabra\-::\-Tamanio (
\begin{DoxyParamCaption}
{}
\end{DoxyParamCaption}
)}}\label{classpalabra_a9b194a7cce2d3713fa278504f09aca9c}


Tamaño de la palabra. 

\begin{DoxyReturn}{Devuelve}
Devuelve el número de términos de la palabra, contando tanto el término en el idioma origen como los términos de su traducción. 
\end{DoxyReturn}
\hypertarget{classpalabra_acef4f5bd0630c89975775218daf913bd}{\index{palabra@{palabra}!Terminos@{Terminos}}
\index{Terminos@{Terminos}!palabra@{palabra}}
\subsubsection[{Terminos}]{\setlength{\rightskip}{0pt plus 5cm}string$\ast$ palabra\-::\-Terminos (
\begin{DoxyParamCaption}
{}
\end{DoxyParamCaption}
)}}\label{classpalabra_acef4f5bd0630c89975775218daf913bd}


Devolución de terminos. 

\begin{DoxyReturn}{Devuelve}
Devuelve un puntero a string que marca el inicio del vector de un término y sus traducciones. La primera posición del vector la ocupa el término en el idioma origen, las restantes son las traducciones del mismo. 
\end{DoxyReturn}


La documentación para esta clase fue generada a partir del siguiente fichero\-:\begin{DoxyCompactItemize}
\item 
include/\hyperlink{palabra_8h}{palabra.\-h}\end{DoxyCompactItemize}

\hypertarget{classtraductor}{\section{Referencia de la Clase traductor}
\label{classtraductor}\index{traductor@{traductor}}
}


T.\-D.\-A. Traductor.  




{\ttfamily \#include $<$traductor.\-h$>$}

\subsection*{Métodos públicos}
\begin{DoxyCompactItemize}
\item 
\hyperlink{classtraductor_ab99562aebb0a518d6e9298bfa9dbadac}{traductor} ()
\begin{DoxyCompactList}\small\item\em Constructor por defecto de la clase. \end{DoxyCompactList}\item 
\hyperlink{classtraductor_a9d3bb37c205eb04dc6f7a0ef6e2d5aa9}{$\sim$traductor} ()
\begin{DoxyCompactList}\small\item\em Destructor de la clase. \end{DoxyCompactList}\item 
\hyperlink{classtraductor_af936ed17ec673c1d48812feaea06eb46}{traductor} (const \hyperlink{classtraductor}{traductor} \&orig)
\begin{DoxyCompactList}\small\item\em Constructor de copia de la clase. \end{DoxyCompactList}\item 
int \hyperlink{classtraductor_a0c07b08c93bd8b8f408716afb6dc0782}{Tamanio} () const 
\begin{DoxyCompactList}\small\item\em Número de palabras del traductor. \end{DoxyCompactList}\item 
\hyperlink{classtraductor}{traductor} \& \hyperlink{classtraductor_ad93a9e15d6ac047bfd2cd5f7f1754d45}{operator=} (const \hyperlink{classtraductor}{traductor} \&trad)
\begin{DoxyCompactList}\small\item\em Operador de asignación de la clase. \end{DoxyCompactList}\item 
void \hyperlink{classtraductor_a7720fe5824d82064b1479ffcc761020e}{Aniade} (\hyperlink{classpalabra}{palabra} w)
\begin{DoxyCompactList}\small\item\em Añade palabras a Traductor. \end{DoxyCompactList}\item 
string $\ast$ \hyperlink{classtraductor_a62c1d8dde134e9122c7172fa6f611e13}{Get\-Traduccion} (string busqueda, int \&nacep)
\begin{DoxyCompactList}\small\item\em Busca las traducciones. \end{DoxyCompactList}\end{DoxyCompactItemize}


\subsection{Descripción detallada}
T.\-D.\-A. Traductor. 

Una instancia {\itshape T} del tipo de datos abstracto {\ttfamily traductor} es un objeto que representa un conjunto de palabras en un idioma concreto y sus posibles traducciones.

Un ejemplo de su uso\-: 
\begin{DoxyCodeInclude}
\textcolor{preprocessor}{#include "\hyperlink{traductor_8h}{traductor.h}"}
\textcolor{preprocessor}{#include "\hyperlink{palabra_8h}{palabra.h}"}
\textcolor{preprocessor}{#include <fstream>}
\textcolor{preprocessor}{#include <iostream>}
\textcolor{keyword}{using namespace }std;
\textcolor{keywordtype}{int} main(\textcolor{keywordtype}{int} argc, \textcolor{keywordtype}{char} * argv[])\{

  \textcolor{keywordflow}{if} (argc!=2)\{
      cout<<\textcolor{stringliteral}{"Dime el nombre del fichero con las traducciones"}<<endl;
      \textcolor{keywordflow}{return} 0;
   \}

   ifstream f (argv[1]);
   \textcolor{keywordflow}{if} (!f)\{
    cout<<\textcolor{stringliteral}{"No puedo abrir el fichero "}<<argv[1]<<endl;
    \textcolor{keywordflow}{return} 0;
   \}
   cout << \textcolor{stringliteral}{"voy a crear traductor"}<< endl ;
   \hyperlink{classtraductor}{traductor} T;
   cout << \textcolor{stringliteral}{"voy a cargar el traductor"}<< endl ;
   f>>T; \textcolor{comment}{//Cargamos en memoria, en el traductor.}
   
   \textcolor{keywordtype}{string} a;
   cout<<\textcolor{stringliteral}{"Dime una palabra en el idioma origen:"};
   getline(cin,a);
   
   \textcolor{keywordtype}{int} nacep=0;
  \textcolor{keywordtype}{string}* trads=T.GetTraduccion(a,nacep);
   
   
   \textcolor{comment}{/* Escrbimos*/}
   cout<<a<<\textcolor{stringliteral}{"-->"};
   \textcolor{keywordflow}{for} (\textcolor{keywordtype}{int} i=0;i<nacep; ++i)
     cout<<trads[i]<<\textcolor{charliteral}{';'};
   cout<<endl;
   
   
   



\}
\end{DoxyCodeInclude}


\begin{DoxyAuthor}{Autor}
Miguel Ángel Torres 
\end{DoxyAuthor}
\begin{DoxyDate}{Fecha}
Octubre 2015 
\end{DoxyDate}


Definición en la línea 56 del archivo traductor.\-h.



\subsection{Documentación del constructor y destructor}
\hypertarget{classtraductor_ab99562aebb0a518d6e9298bfa9dbadac}{\index{traductor@{traductor}!traductor@{traductor}}
\index{traductor@{traductor}!traductor@{traductor}}
\subsubsection[{traductor}]{\setlength{\rightskip}{0pt plus 5cm}traductor\-::traductor (
\begin{DoxyParamCaption}
{}
\end{DoxyParamCaption}
)}}\label{classtraductor_ab99562aebb0a518d6e9298bfa9dbadac}


Constructor por defecto de la clase. 

\begin{DoxyReturn}{Devuelve}
Crea un objeto traductor con ninguna defenición. Es necesario usar el operador $>$$>$ para cargar definiciones. 
\end{DoxyReturn}
\hypertarget{classtraductor_a9d3bb37c205eb04dc6f7a0ef6e2d5aa9}{\index{traductor@{traductor}!$\sim$traductor@{$\sim$traductor}}
\index{$\sim$traductor@{$\sim$traductor}!traductor@{traductor}}
\subsubsection[{$\sim$traductor}]{\setlength{\rightskip}{0pt plus 5cm}traductor\-::$\sim$traductor (
\begin{DoxyParamCaption}
{}
\end{DoxyParamCaption}
)}}\label{classtraductor_a9d3bb37c205eb04dc6f7a0ef6e2d5aa9}


Destructor de la clase. 

\begin{DoxyReturn}{Devuelve}
Destruye el objeto traductor 
\end{DoxyReturn}
\hypertarget{classtraductor_af936ed17ec673c1d48812feaea06eb46}{\index{traductor@{traductor}!traductor@{traductor}}
\index{traductor@{traductor}!traductor@{traductor}}
\subsubsection[{traductor}]{\setlength{\rightskip}{0pt plus 5cm}traductor\-::traductor (
\begin{DoxyParamCaption}
\item[{const {\bf traductor} \&}]{orig}
\end{DoxyParamCaption}
)}}\label{classtraductor_af936ed17ec673c1d48812feaea06eb46}


Constructor de copia de la clase. 


\begin{DoxyParams}{Parámetros}
{\em orig} & Traductor que se va a copiar para construir el nuevo objeto \\
\hline
\end{DoxyParams}
\begin{DoxyReturn}{Devuelve}
Crea un traductor con las mismas palabras que orig 
\end{DoxyReturn}


\subsection{Documentación de las funciones miembro}
\hypertarget{classtraductor_a7720fe5824d82064b1479ffcc761020e}{\index{traductor@{traductor}!Aniade@{Aniade}}
\index{Aniade@{Aniade}!traductor@{traductor}}
\subsubsection[{Aniade}]{\setlength{\rightskip}{0pt plus 5cm}void traductor\-::\-Aniade (
\begin{DoxyParamCaption}
\item[{{\bf palabra}}]{w}
\end{DoxyParamCaption}
)}}\label{classtraductor_a7720fe5824d82064b1479ffcc761020e}


Añade palabras a Traductor. 


\begin{DoxyParams}{Parámetros}
{\em w} & objeto del T\-D\-A palabra que se va a adherir a traductor \\
\hline
\end{DoxyParams}
\hypertarget{classtraductor_a62c1d8dde134e9122c7172fa6f611e13}{\index{traductor@{traductor}!Get\-Traduccion@{Get\-Traduccion}}
\index{Get\-Traduccion@{Get\-Traduccion}!traductor@{traductor}}
\subsubsection[{Get\-Traduccion}]{\setlength{\rightskip}{0pt plus 5cm}string$\ast$ traductor\-::\-Get\-Traduccion (
\begin{DoxyParamCaption}
\item[{string}]{busqueda, }
\item[{int \&}]{nacep}
\end{DoxyParamCaption}
)}}\label{classtraductor_a62c1d8dde134e9122c7172fa6f611e13}


Busca las traducciones. 


\begin{DoxyParams}{Parámetros}
{\em busqueda} & Palabra que se va a intentar traducir \\
\hline
{\em nacep} & Número de traducciones encontradas para busqueda \\
\hline
\end{DoxyParams}
\begin{DoxyReturn}{Devuelve}
Devuelve un puntero a string con el vector formado por la palabra buscada (0) y las traducciones (1-\/nacep) 
\end{DoxyReturn}
\begin{DoxyPrecond}{Precondición}
nacep es un parametro pasado por referencia, se perderá su valor para dar el número de traduccioens de busqueda 
\end{DoxyPrecond}
\hypertarget{classtraductor_ad93a9e15d6ac047bfd2cd5f7f1754d45}{\index{traductor@{traductor}!operator=@{operator=}}
\index{operator=@{operator=}!traductor@{traductor}}
\subsubsection[{operator=}]{\setlength{\rightskip}{0pt plus 5cm}{\bf traductor}\& traductor\-::operator= (
\begin{DoxyParamCaption}
\item[{const {\bf traductor} \&}]{trad}
\end{DoxyParamCaption}
)}}\label{classtraductor_ad93a9e15d6ac047bfd2cd5f7f1754d45}


Operador de asignación de la clase. 


\begin{DoxyParams}{Parámetros}
{\em trad} & Traductor que se va a asignar. \\
\hline
\end{DoxyParams}
\begin{DoxyReturn}{Devuelve}
Convierte el traductor en una copia de trad 
\end{DoxyReturn}
\hypertarget{classtraductor_a0c07b08c93bd8b8f408716afb6dc0782}{\index{traductor@{traductor}!Tamanio@{Tamanio}}
\index{Tamanio@{Tamanio}!traductor@{traductor}}
\subsubsection[{Tamanio}]{\setlength{\rightskip}{0pt plus 5cm}int traductor\-::\-Tamanio (
\begin{DoxyParamCaption}
{}
\end{DoxyParamCaption}
) const}}\label{classtraductor_a0c07b08c93bd8b8f408716afb6dc0782}


Número de palabras del traductor. 

\begin{DoxyReturn}{Devuelve}
Devuelve el número total de palabras que contiene el traductor en la actualidad 
\end{DoxyReturn}


La documentación para esta clase fue generada a partir del siguiente fichero\-:\begin{DoxyCompactItemize}
\item 
include/\hyperlink{traductor_8h}{traductor.\-h}\end{DoxyCompactItemize}

\chapter{Documentación de archivos}
\hypertarget{palabra_8h}{\section{Referencia del Archivo include/palabra.h}
\label{palabra_8h}\index{include/palabra.\-h@{include/palabra.\-h}}
}


Fichero cabecera del T\-D\-A palabra.  


\subsection*{Clases}
\begin{DoxyCompactItemize}
\item 
class \hyperlink{classpalabra}{palabra}
\begin{DoxyCompactList}\small\item\em T.\-D.\-A. Palabra. \end{DoxyCompactList}\end{DoxyCompactItemize}


\subsection{Descripción detallada}
Fichero cabecera del T\-D\-A palabra. 

Definición en el archivo \hyperlink{palabra_8h_source}{palabra.\-h}.


\hypertarget{traductor_8h}{\section{Referencia del Archivo include/traductor.h}
\label{traductor_8h}\index{include/traductor.\-h@{include/traductor.\-h}}
}


Fichero cabecera del T\-D\-A Traductor.  


{\ttfamily \#include $<$string$>$}\\*
{\ttfamily \#include \char`\"{}palabra.\-h\char`\"{}}\\*
\subsection*{Clases}
\begin{DoxyCompactItemize}
\item 
class \hyperlink{classtraductor}{traductor}
\begin{DoxyCompactList}\small\item\em T.\-D.\-A. Traductor. \end{DoxyCompactList}\end{DoxyCompactItemize}
\subsection*{Funciones}
\begin{DoxyCompactItemize}
\item 
istream \& \hyperlink{traductor_8h_aa9bc39697cda076e62ed23bec55ea18e}{operator$>$$>$} (istream \&f, \hyperlink{classtraductor}{traductor} \&datos)
\begin{DoxyCompactList}\small\item\em Cargar traductor a partir de un flujo de datos. \end{DoxyCompactList}\end{DoxyCompactItemize}


\subsection{Descripción detallada}
Fichero cabecera del T\-D\-A Traductor. 

Definición en el archivo \hyperlink{traductor_8h_source}{traductor.\-h}.



\subsection{Documentación de las funciones}
\hypertarget{traductor_8h_aa9bc39697cda076e62ed23bec55ea18e}{\index{traductor.\-h@{traductor.\-h}!operator$>$$>$@{operator$>$$>$}}
\index{operator$>$$>$@{operator$>$$>$}!traductor.h@{traductor.\-h}}
\subsubsection[{operator$>$$>$}]{\setlength{\rightskip}{0pt plus 5cm}istream\& operator$>$$>$ (
\begin{DoxyParamCaption}
\item[{istream \&}]{f, }
\item[{{\bf traductor} \&}]{datos}
\end{DoxyParamCaption}
)}}\label{traductor_8h_aa9bc39697cda076e62ed23bec55ea18e}


Cargar traductor a partir de un flujo de datos. 


\begin{DoxyParams}{Parámetros}
{\em f} & Flujo desde el que se van a introducir las palabras \\
\hline
{\em datos} & Traductor al que se van a cargar las palabras. \\
\hline
\end{DoxyParams}
\begin{DoxyReturn}{Devuelve}
Devuelve el flujo para permitir encadenar varios flujos 
\end{DoxyReturn}
\begin{DoxyPrecond}{Precondición}
En el flujo, las palabras deberán estar agrupadas de la siguiente forma \-: palabra\-\_\-origen;traducción;traducción;traducción....\par
 E\-S I\-M\-P\-R\-E\-S\-C\-I\-N\-D\-I\-B\-L\-E Q\-U\-E E\-L D\-O\-C\-U\-M\-E\-N\-T\-O T\-E\-R\-M\-I\-N\-E C\-O\-N U\-N R\-E\-T\-O\-R\-N\-O D\-E C\-A\-R\-R\-O 
\end{DoxyPrecond}

%--- End generated contents ---

% Index
\newpage
\phantomsection
\addcontentsline{toc}{chapter}{Índice}
\printindex

\end{document}
